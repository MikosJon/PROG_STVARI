\documentclass[a4paper]{article}
\usepackage{amsmath, amssymb, xifthen, xcolor, calc, printlen}
\usepackage[slovene]{babel}
\usepackage[T1]{fontenc}
\usepackage[utf8]{inputenc}
\usepackage{lmodern}

\newcommand{\numR}{\mathbb{R}}
\newcommand{\numC}{\mathbb{C}}
\newcommand{\numN}{\mathbb{N}}

\newcommand{\cont}{\mathcal{C}}

\newcommand{\continousOn}[1]{\cont(#1)}

\newcommand{\vect}[1]{\mathbf{#1}}

\newcommand{\smartCont}[1][]{
    \cont%
    \ifthenelse{\equal{#1}{}}
        {}
        {(#1)}
}

\newcommand{\cmd}[1]{\textcolor{blue}{\texttt{\textbackslash#1}}}

\newcommand{\demolength}[1]{
    \rule{.1pt}{.5em}%
    \rule{#1}{.1pt}%
    \rule{.1pt}{.5em}
}

\newcommand{\whitespaceLength}[1]{\demolength{\widthof{#1}}}

\begin{document}
\Large
\begin{enumerate}
    \item Number systems: $\numR, \numC, \numN$
    \item Continous functions: $\cont$
    \item Continous on $[0,1]$: $\continousOn{[0,1]}$
    \item Vector v: $\vect{v}$
    \item Continous on $\numR^2$: $\smartCont[\numR^2]$
    \item Command $\cmd{cont}$ gives continous functions
    \item A length of 1em: \demolength{1em}
    \item Widths of spaces:
    \begin{itemize}
        \item Prof. A. Bauer = Prof. A. Bauer;
        \item Prof.~A.~Bauer = Prof.\texttt{\textasciitilde}A.\texttt{\textasciitilde}Bauer;
        \item Prof.\ A.\ Bauer = Prof.\texttt{\textbackslash} A.\texttt{\textbackslash} Bauer;
        \item Prof.\,A.\,Bauer = Prof.\texttt{\textbackslash},A.\texttt{\textbackslash,}Bauer;
    \end{itemize}
    \texttt{\textasciitilde} is an unbreakable space;
    \texttt{\textbackslash}  is a forced normal sized space;
    \texttt{\textbackslash}, is a thin sized unbreakable space
\end{enumerate}
\end{document}