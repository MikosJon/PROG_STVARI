\documentclass[a4paper,12pt]{article}
\usepackage[slovene]{babel}
\usepackage[utf8]{inputenc}
\usepackage[T1]{fontenc}
\usepackage{lmodern}

\usepackage{amsmath, amssymb}

\newcommand{\numC}{\mathbb{C}}
\newcommand{\numR}{\mathbb{R}}
\newcommand{\numZ}{\mathbb{Z}}
\newcommand{\numN}{\mathbb{N}}

\pagestyle{empty}

\begin{document}
\section*{Naloge iz matematike}
\begin{enumerate}
\item
Dokaži, da je enačba $(P \cap X) \cup (Q \cap X^c) = \emptyset$
rešljiva natanko tedaj, ko je $Q \subseteq P^c$.

\item
Pokaži:
\begin{itemize}
    \item $M = N \iff M + N = \emptyset$
    \item $M = N = \emptyset \iff M \cup N = \emptyset$
\end{itemize}

\item
Ali obstaja tak izjavni izraz $A$ da bosta izraza
$(p \land A) \lor (p \Rightarrow \neg A)$ in $(p \Rightarrow A) \Rightarrow q$
enakovredna?

\item
Dokaži:
\begin{itemize}
    \item $(A \Rightarrow B) \sim (\neg B \Rightarrow \neg A)$
    \item $\neg (A \lor B) \sim \neg A \land \neg B$
\end{itemize}

\item
Poišči preneksno obliko formule $\exists x : P(x) \land \forall x :
Q(x) \Rightarrow \forall x : R(x)$.

\item
Vektorja $\vec{c} = \vec{a} + 2\vec{b}$ in $\vec{d} = \vec{a} - \vec{b}$
sta pravokotna in imata dolžino 1. Določi kot med vektorjema $\vec{a}$ in $\vec{b}$.

\item
Določi definicijsko območje funkcije
\[
    f(x) = \log \frac{x^2 + 1}{x^2 - 4x + 3}
\]

\item
Izračunaj
\[
    \cos^2 \frac{3\pi}{8} + \cos^2 \frac{5\pi}{8} + 
    \cos^2 \frac{7\pi}{8} + \cos^2 \frac{8\pi}{8}
\]

\item
Dokaži, da za vsa naravna števila $n$ velja
\[
    \frac{1}{\sqrt{1}} + \frac{1}{\sqrt{2}} + \cdots + \frac{1}{\sqrt{n}} \geq \sqrt{n}  
\]

\item
Naj bo $z$ kompleksno število, $z \neq 1$ in $|z| = 1$.
Dokaži, da je število $i \frac{z+1}{z-1}$ realno.

\item
Pokaži, da je funkcija $x \mapsto \sqrt{x}$ enakomerno zvezna na $[0, \infty)$.

\item
Izračunaj limito
\[
    \lim_{x \to \infty} (\sin \sqrt{x + 1} - \sin \sqrt{x})    
\]

\item
Dani sta grupi $(G,\ast)$ in $(H, \circ)$. V množici $G \times H$ definiramo operacijo
\[
    (g_1, h_1) \cdot (g_2, h_2) = (g_1 \ast g_2, h_1 \circ h_2)
\]
Pokaži, da je množica $G \times H$ grupa za to operacijo.

\item
Pokaži, da ima $f(x) = 3x + \sin(2x)$ inverzno funkcijo in izračunaj $(f^-1)(3\pi)$.

\item
Izračunaj integral korenske funkcije
\[
    \int \frac{2 + \sqrt{x + 1}}{(x + 1)^2 - \sqrt{x + 1}}\,dx    
\]

\item
Krivulja je podana parametrično z enačbama
\[
    x(t) = \int_{1}^{t} \frac{\sin u}{u^2}\,du \qquad
    y(t) = \int_{1}^{t} \frac{\cos u}{u^2}\,du
\]
Izračunaj dolžino poti od točke $x = 0$ do točke, v kateri je tangenta prvič navpična.

\item
Naj bo $\sum_{n = 1}^{\infty} a_n$ absolutno konvergentna vrsta in $a_n \neq 1$ za $n \in \numR$.
Dokaži, da sta vrsti
\[
    \sum_{n = 1}^{\infty} \frac{a_n}{1 + a_n} \qquad \text{in} \qquad
    \sum_{n = 1}^{\infty} \frac{a_{n}^{2}}{1 + a_{n}^{2}}
\]
absolutno konvergentni.

\item
Funkcijsko zaporedje $f_n : [a, b] \to [c, d]$ enakomerno konvergira na $[a, b]$ proti
funkciji $f$. Naj bo $g : [c, d] \to \numR$ zvezna. Dokaži, da funkcijsko zaporedje
$g \circ f_n$ enakomerno konvergina na $[a, b]$ in določi njegovo limitno funkcijo.

\item
Izračunaj limito zaporedja
\[
    \lim_{n \to \infty} \frac{\sqrt[3]{n^2 + n -1} + \sqrt[3]{n} + n^2}{2n^2 + \sqrt{n} + 1}
\]

\item
Izračunaj
$
\begin{pmatrix}
    1 & 2 & 3 & 4 & 5 & 6\\
    4 & 5 & 2 & 6 & 3 & 1
\end{pmatrix}^{-2000}
$

\item
Poenostavi
\[
    \frac{
        \dfrac{3 + i}{2 - 2i} + \dfrac{7i}{1-i}
    }{
        1 + \dfrac{i-1}{4} - \dfrac{5}{2-3i}
    }    
\]

\item
Za dani zaporedji preveri, ali sta konvergentni
\[
    a_n = \underbrace{\sqrt{2 + \sqrt{2 + \cdots + \sqrt{2}}}}_{\text{$n$ korenov}} \qquad
    b_n = \underbrace{\sin(\sin(\dots(\sin1)\dots))}_{\text{$n$ sinusov}}   
\]

\item
Ugotovi, ali obstaja
\[
    \lim_{y \to 0} y \left(\frac{y + 1}{y} - \sqrt{\frac{y^2 + 1}{y^2}}\right)    
\]
Pomagaj si z limitama funkcije $\dfrac{x + 1 - \sqrt{x^2 + 1}}{x}$ v $-\infty$ in $\infty$.

\item
Izračunaj naslednjo determinanto $2n \times 2n$, ki ima na neoznačenih mestih ničle.
\[
\begin{vmatrix}
    1 &   &        &     &  1     &     &     &        &    \\
      & 2 &        &     &  2     &     &     &        &    \\
      &   & \ddots &     & \vdots &     &     &        &    \\
      &   &        & n-1 & n-1    &     &     &        &    \\
    1 & 2 & \dots  & n-1 & n      & n+1 & n+2 & \dots  & 2n \\
      &   &        &     & n+1    & n+1 &     &        &    \\
      &   &        &     & n+2    &     & n+2 &        &    \\
      &   &        &     & \vdots &     &     & \ddots &    \\
      &   &        &     & 2n     &     &     &        & 2n
\end{vmatrix}
\]

\item
Dana je funkcija
\[
    f(x, y) = 
    \begin{cases}
        \frac{3x^{2}y - y^3}{x^2 + y^2}; & (x, y) \neq (0, 0) \\
        a;                               & (x, y) = (0, 0)
    \end{cases}    
\]
\begin{itemize}
    \item Določi parameter $a$ tako, da bo $f$ zvezna.
    \item Izračunaj parcialna odvoda $f_x(x, y)$ in $f_y(x, y)$ za $(x, y) \neq (0, 0)$.
    \item Izračunaj parcialna odvoda $f_x(0, 0)$ in $f_y(0, 0)$. Če obstaja, izračunaj limito
    \[
        \lim_{(x, y) \to (0, 0)} \frac{f(x, y) - f_x(0, 0) - f_y(0, 0)}{\sqrt{x^2 + y^2}}
    \]
    Ali je funkcija $f$ diferenciabilna?
\end{itemize}

\item
Poišči vse rešitve enačbe
\begin{multline*}
    (1 + x + x^2) \cdot (1 + x + x^2 + x^3 + \dots + x^9 + x^{10}) =\\
    = (1 + x + x^2 + x^3 + x^4 + x^5 + x^6)^2
\end{multline*}

\item
Dokaži \emph{binomsko formulo}: za vsaki realni števili $a$ in $b$
in za vsako naravno število $n$ velja
\begin{align*}
    (a + b)^n &= a^n + na^{n-1}b + \cdots + \binom{n}{k}a^{n-k} b^k + \cdots nab^{n-1} + b^n\\
              &= \sum_{k = 0}^{n} \binom{n}{k} a^{n-k} b^k
\end{align*}

\item
Naj bo
\begin{align*}
    G &= \{z \in \numC; z = 2^k (\cos(m \pi \sqrt{2}) + i\sin(m \pi \sqrt{2})),k,m \in \numZ\}\\
    H &= \{(x, y) \in \numR^2; x, y \in \numZ\}
\end{align*}
\begin{enumerate}
    \item Pokaži, da je $G$ podgrupa v grupi $(\numC \backslash \{0\}, \cdot)$
    neničelnih kompleksnih števil za običajno množenje.
    \item Pokaži, da je $H$ podgrupa v aditivni grupi $(\numR^2, +)$
    ravninskih vektorjev za običajno seštevanje po komponentah.
    \item Pokaži, da je preslikava $f: H \to G$, podana s pravilom
    \[
        (x, y) \mapsto 2^x  (\cos(y \pi \sqrt{2}) + i\sin(y \pi \sqrt{2}))
    \]
    izomorfizem grup $G$ in $H$.
\end{enumerate}

\item
Nariši grafe funkcij:
\begin{align*}
    y &= x^2 - 3|x| + 2    & y &= 3 \sin(\pi + x) - 2\\
    y &= \log_2(x - 2) + 3 & y &= 2 \sqrt{x^2 + 15} + 6\\
    y &= 2^{x - 3} + 1     & y &= \cos(x - 3) + \sin^2(x + 1) 
\end{align*}
\end{enumerate}
\end{document}
