\documentclass[a4paper,12pt]{article}
\usepackage[slovene]{babel}
\usepackage[utf8]{inputenc}
\usepackage[T1]{fontenc}
\usepackage{lmodern}
\pagestyle{empty}

\begin{document}
\section*{Naloge iz matematike}
\begin{enumerate}
\item
Dokaži, da je enačba ??
rešljiva natanko tedaj, ko je ??.

\item
Pokaži:
??
??

\item
Ali obstaja tak izjavni izraz ??, da bosta izraza
?? in ??
enakovredna?

\item
Dokaži:
??
??

\item
Poišči preneksno obliko formule ??.

\item
Vektorja ?? in ??
sta pravokotna in imata dolžino 1. Določi kot med vektorjema ?? in ??.

\item
Določi definicijsko območje funkcije
??

\item
Izračunaj
??

\item
Dokaži, da za vsa naravna števila $n$ velja
??

\item
Naj bo ?? kompleksno število, ?? in ??.
Dokaži, da je število ?? realno.

\item
Pokaži, da je funkcija ?? enakomerno zvezna na ??.

\item
Izračunaj limito
??

\item
Dani sta grupi ??. V množici ?? definiramo operacijo
??
Pokaži, da je množica ?? grupa za to operacijo.

\item
Pokaži, da ima ?? inverzno funkcijo in izračunaj ??.

\item
Izračunaj integral korenske funkcije
??

\item
Krivulja je podana parametrično z enačbama
??
Izračunaj dolžino poti od točke ?? do točke, v kateri je tangenta prvič navpična.

\item
Naj bo ?? absolutno konvergentna vrsta in ?? za ??.
Dokaži, da sta vrsti
??
absolutno konvergentni.

\item
Funkcijsko zaporedje ?? enakomerno konvergira na ?? proti funkciji ??.
Naj bo ?? zvezna. Dokaži, da funkcijsko zaporedje ??
enakomerno konvergina na ?? in določi njegovo limitno funkcijo.

\item
Izračunaj limito zaporedja
??

\item
Izračunaj
??

\item
Poenostavi
??

\item
Za dani zaporedji preveri, ali sta konvergentni
??

\item
Ugotovi, ali obstaja
??
Pomagaj si z limitama funkcije ?? v ?? in ??.

\item
Izračunaj naslednjo determinanto ??, ki ima na neoznačenih mestih ničle.
??

\item
Dana je funkcija
??
Določi parameter ?? tako, da bo ?? zvezna.
Izračunaj parcialna odvoda ?? in ?? za ??.
Izračunaj parcialna odvoda ?? in ??.
Če obstaja, izračunaj limito
??
Ali je funkcija ?? diferenciabilna?

\item
Poišči vse rešitve enačbe
??

\item
Dokaži binomsko formulo: za vsaki realni števili ?? in ?? in za vsako naravno število ?? velja
??

\item
Naj bo
??
Pokaži, da je ?? podgrupa v grupi ??
neničelnih kompleksnih števil za običajno množenje.
Pokaži, da je ?? podgrupa v aditivni grupi ??
ravninskih vektorjev za običajno seštevanje po komponentah.
Pokaži, da je preslikava ??, podana s pravilom
??
izomorfizem grup ?? in ??.

\item
Nariši grafe funkcij:
??
\end{enumerate}
\end{document}
